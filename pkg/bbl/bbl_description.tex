%%
%%  $Header: /u/gcmpack/MITgcm/pkg/bbl/bbl_description.tex,v 1.1 2011/08/06 03:13:22 dimitri Exp $
%%  $Name:  $
%%

Package ``BBL'' is a simple bottom boundary layer scheme.  The initial
motivation is to allow dense water that forms on the continental shelf around
Antarctica (High Salinity Shelf Water) in the CS510 configuration to sink to
the bottom of the model domain and to become a source of Antarctic Bottom
Water.  The bbl package aims to address the following two crippling limitations of
package down_slope:

(i) In pkg/down_slope, dense water cannot flow down-slope unless there is a
step, i.e., a change of vertical level in the bathymetry.  In pkg/bbl, dense
water can flow down-slope even on a slightly inclined or flat bottom.

(ii) In pkg/down_slope, dense water is diluted as it flows into grid cells
whose thickness depends on the model's vertical grid and that are typically
much thicker than the bottom boundary layer.  In pkg/bbl, the dense water is
contained in a thinner layer and hence better able to preserve its tracer
properties.

Specifically, the bottommost wet grid cell of thickness

         Thk = hFacC(kBot) * drF(kBot),

of tracer properties Tracer, and of density rho is divided in two sub-levels:

1. A bottom boundary layer with T/S tracer properties bbl_Tracer, density
bbl_rho, thickness bbl_eta, and volume bbl_Volume.

2. A residual thickness

         resThk = Thk - bbl_eta

with tracer properties

         resTracer = ( Tracer * Thk - bbl_Tracer * bbl_eta ) / resThk

such that the volume integral of bbl_Tracer and resTracer is consistent with
the Tracer properties of bottommost wet grid cell.

At every time step, the bottom boundary layer properties bbl_Tracer evolve as
follows:

I. There is a vertical exchange between the BBL and the residual volume of
bottommost wet grid cell:

(i) If rho >= bbl_rho then set bbl_Tracer = Tracer

(ii) If bbl_rho > rho , the T/S properties of the BBL diffuse into the
residual volume with a relaxation time scale of bbl_RelaxR seconds.

         bbl_Tracer(T+deltaT) = bbl_Tracer(T) +
                                deltaT * (resTheta-bbl_Tracer(T)) / bbl_RelaxR

The above two operations do not change the tracer properties of the bottommost
wet grid box.  They only redistribute properties between bbl_eta and resThk.

II. There is horizontal exchange between adjacent bottom boundary layer cells
when heavy BBL water is above or at the same level as lighter BBL water.  The
strength of the horizontal exchange is controlled by time scale bbl_RelaxH:

(i) First this horizontal exchange is accumulated in BBL tracer tendency terms
zonally and meridionally:

         bbl_TendTracer(i) = bbl_TendTracer(i) +
                             ( bbl_Tracer(i+1) - bbl_Tracer(i) ) /
                             bbl_RelaxH

         bbl_TendTracer(i+1) = bbl_TendTracer(i+1) -
                             ( bbl_Tracer(i+1) - bbl_Tracer(i) ) * bbl_Volume(i) /
                             ( bbl_Volume(i+1) * bbl_RelaxH )

(ii) Then these tendency terms are applied to the BBL tracer properties:

         bbl_Tracer(T+deltaT) = bbl_Tracer(T) + deltaT * bbl_TendTracer

(iii) Finally these tracer tendencies are scaled by the full thickness Thk of
the bottommost wet cell:

         bbl_TendTracerScaled = bbl_TendTracer * bbl_eta / Thk

and applied to the model's tracer quantities by bbl_tendency_apply.  Apart
from this lateral exchange of tracer properties between the bottommost model
grid cells, all other normal advection diffusion terms are also applied.
