%%
%%  $Header: /u/gcmpack/MITgcm/pkg/seaice/seaice_description.tex,v 1.2 2004/05/05 22:50:54 dimitri Exp $
%%  $Name:  $
%%

\chapter{Dynamic Thermodynamic Seaice Package}
  
Package ``seaice'' provides a dynamic and thermodynamic interactive sea-ice
model.  Sea-ice model thermodynamics are based on Hibler \cite{hib80}, that
is, a 2-category model that simulates ice thickness and concentration.  Snow
is simulated as per Zhang et al. \cite{zha98a}.  Although recent years have
seen an increased use of multi-category thickness distribution sea-ice models
for climate studies, the Hibler 2-category ice model is still the most widely
used model and has resulted in realistic simulation of sea-ice variability on
regional and global scales.  Being less complicated, compared to
multi-category models, the 2-category model permits easier application of
adjoint model optimization methods.

Note, however, that the Hibler 2-category model and its variants use a
so-called zero-layer thermodynamic model to estimate ice growth and decay.
The zero-layer thermodynamic model assumes that ice does not store heat and,
therefore, tends to exaggerate the seasonal variability in ice thickness.
This exaggeration can be significantly reduced by using Semtner's \cite{sem76}
three-layer thermodynamic model that permits heat storage in ice.  Recently,
the three-layer thermodynamic model has been reformulated by Winton
\cite{win00}.  The reformulation improves model physics by representing the
brine content of the upper ice with a variable heat capacity.  It also
improves model numerics and consumes less computer time and memory.  The
Winton sea-ice thermodynamics have been ported to the MIT GCM; they currently
reside under pkg/thsice.  At present pkg/thsice is not fully compatible with
pkg/seaice and with pkg/exf.  But the eventual objective is to have fully
compatible and interchangeable thermodynamic packages for sea-ice, so that it
becomes possible to use Hibler dynamics with Winton thermodyanmics.

The ice dynamics models that are most widely used for large-scale
climate studies are the viscous-plastic (VP) model \cite{hib79}, the
cavitating fluid (CF) model \cite{fla92}, and the
elastic-viscous-plastic (EVP) model \cite{hun97}.  Compared to the VP
model, the CF model does not allow ice shear in calculating ice
motion, stress, and deformation.  EVP models approximate VP by adding
an elastic term to the equations for easier adaptation to parallel
computers.  Because of its higher accuracy in plastic solution and
relatively simpler formulation, compared to the EVP model, we decided
to use the VP model as the dynamic component of our ice model.  To do
this we extended the alternating-direction-implicit (ADI) method of
Zhang and Rothrock \cite{zha00} for use in a parallel configuration.

The sea ice model requires the following input fields: 10-m winds, 2-m air
temperature and specific humidity, downward longwave and shortwave radiations,
precipitation, evaporation, and river and glacier runoff.  The sea ice model
also requires surface temperature from the ocean model and third level
horizontal velocity which is used as a proxy for surface geostrophic
velocity.  Output fields are surface wind stress, evaporation minus
precipitation minus runoff, net surface heat flux, and net shortwave flux.
The sea-ice model is global: in ice-free regions bulk formulae are used to
estimate oceanic forcing from the atmospheric fields.

% This is a (commented out) example of using epsfig to
% include a figure, Fig.~\ref{overturn}.
% \begin{figure}[t]
% {\epsfig{file=../../pkg/seaice/seaice_fig1.eps,width=3.2in}}
% \caption{Impact of Sea Ice on Ocean Circulation.}
% \label{overturn}
% \end{figure}
