\documentclass[12pt]{article}
\usepackage[a4paper, total={6in, 8in}]{geometry}
\usepackage{amsmath,bm}
\usepackage[round,comma]{natbib}
\title{Maxwell-Elasto-Brittle rheology: Implementation in MITgcm}
\author{Martin Losch, \ldots}
\newcommand{\vek}[1]{\ensuremath{\vec{\bm{#1}}}}
\newcommand{\vtau}{\vek{\bm{\tau}}}
\newcommand{\sr}{\dot{\epsilon}}
\newcommand{\srb}{\dot{\bm{\epsilon}}}
\newcommand{\vsr}{\dot{\mathbf{\epsilon}}}
\begin{document}
\maketitle

\begin{itemize}
\item momentum equations
  \begin{align}
  \label{eq:momseaice}
  m \frac{D\vek{u}}{Dt} &= -mf\vek{k}\times\vek{u} + \vtau_{air} +
  \vtau_{ocean} - m \vec{\nabla}{\phi(0)} + \vec{\nabla}\cdot\bm{\sigma},
  \end{align}
  with $m=\rho\,h\,c$, $\rho$=density, $h$=thickness, $c$=concentration (sea ice fraction), etc., $H=h\,c$
\item VP-rheology (Reiner-Riplin relation):
  \begin{align}
    \label{eq:vp}\sigma_{ij} &= 2 \eta \sr_{ij} +\left[(\zeta-\eta)\sr_{kk}-\frac{P}{2}\right]\delta_{ij}\\
                             &= \left[\bm{K}(\zeta-\eta,\eta):\srb
                               \right]_{ij} - \frac{P}{2}\delta_{ij}\\
    \intertext{with}
    \zeta &= \frac{P}{2\Delta} \\
    \eta &= \frac{\zeta}{e^2} = \frac{P}{2\Delta e^2} \\
    \intertext{MEB-equations}
    \frac{1}{E}\frac{D\bm{\sigma}}{Dt} + \frac{\bm{\sigma}}{\eta_{v}}
                &= \left[\bm{K}(\Lambda,M):\srb\right] \\
    \leftrightarrow \qquad
    \frac{D\bm{\sigma}}{Dt} + \frac{E}{\eta_{v}}\bm{\sigma}
                &= \frac{D\bm{\sigma}}{Dt} + \lambda(t)\bm{\sigma}
                  = E\left[\bm{K}(\Lambda,M):\srb\right] \\
    \intertext{with the Elasticity $E=E(t)$, viscosity $\eta_{v}=\eta_{v}(t)$, reciprocal of the relaxation time scale $\lambda(t)=\frac{E(t)}{\eta_{v}(t)} = \tau_r^{-1}(t)$, and}
    \label{eq:meb}\left(\bm{K}(\Lambda,M):\srb\right)_{ij}
                &= \frac{\nu}{(1+\nu)(1-\nu)}\sr_{kk}\delta_{ij}
                  + 2\,\frac{1}{2(1+\nu)}\sr_{ij} \\
                & = \Lambda\sr_{kk}\delta_{ij} +
                  2M\sr_{ij} \\
  \end{align}
  with the Poisson ratio $\nu = 0.3$, and the Lam{\'e} coefficients $\Lambda$ and $M$ for planar 2D stress [reference]. By comparing Eq.\,(\ref{eq:vp}) to (\ref{eq:meb}), we can see the following anaologies:
  \begin{align}
    \label{eq:p0}     P &\rightarrow0 \quad\text{(PRESS)}\\
    \label{eq:lambda} E\Lambda &\rightarrow \zeta - \eta \\
    \label{eq:mu}     EM
    = \frac{E}{2(1+\nu)} &\rightarrow \eta \quad\text{(ETA)} \\
    \intertext{so that}
    \label{eq:zeta} \text{(ZETA:)}\qquad
    \zeta &\leftarrow E(\Lambda+M) \\
                &= E\left\{\frac{\nu}{(1+\nu)(1-\nu)} +
                  \frac{1}{2(1+\nu)}\right\} \nonumber \\
                &= \frac{E\,(3\nu+1)}{2(1+\nu)(1-\nu)} \nonumber \\
    \text{(PRESS0:)}\qquad E &= E_0\,d\,H\,\exp\left[-C^*(1-c)\right] \\
    E_0&=\text{SEAICE\_strength (runtime parameter)}
  \end{align}
\item we introduce the following abbreviations and conventions:
  \begin{align}
    \sr_{+}&=\sr_{11}+\sr_{22}\\
    \sr_{-}&=\sr_{11}-\sr_{22}\\
    \sigma_{+}&=\sigma_{11}+\sigma_{22}\\
    \sigma_{-}&=\sigma_{11}-\sigma_{22} \\
    \intertext{with the abbreviations we have for the VP equations}
    \sigma_{+} &= 2\,\zeta\,\sr_{+} - P \\
    \sigma_{-} &= 2 \,\eta\,\sr_{-} = 2\,(\zeta/e^2)\,\sr_{-}\\
    \intertext{and for the MEB equations}
    \label{eq:mebplus} E\left(\bm{K}:\srb\right)_{+} &= 2\,\zeta\,\sr_{+}
    = 2\,(\Lambda+M)\,\sr_{+}\\
    \label{eq:mebminus}E\left(\bm{K}:\srb\right)_{-} &= 2\, \eta\,\sr_{-}
    = 2\,M\,\sr_{-}\\
    \intertext{principle stress components:}
    \sigma_1 &=\frac{\sigma_{11}+\sigma_{22}}{2}
               + \sqrt{\frac{(\sigma_{11}-\sigma_{22})^2}{2}+\sigma_{12}}\\
    \sigma_2 &=\frac{\sigma_{11}+\sigma_{22}}{2}
               - \sqrt{\frac{(\sigma_{11}-\sigma_{22})^2}{2}+\sigma_{12}}\\
    \intertext{stress invariants:}
    \text{divergence:}\qquad\sigma_{I} &= \sigma_1+\sigma_2
                                         = \sigma_{11}+\sigma_{22}\\
    \text{shear:}\qquad\sigma_{II} &= \sigma_1-\sigma_2=
                                    \sqrt{(\sigma_{11}-\sigma_{22})^2+4\sigma_{12}^{2}}
                   \\
  \end{align}
  and similar for $\sr_{ij}$.
\item some parameters:
  \begin{alignat}{1}
    E_0&\approx 10^{9}\text{\,N\,m$^{-2}$}
    \quad\text{elasticity modulus of undamaged ice} \\
    \lambda_{0} &\approx 10^{-7}\text{\,s}
    \quad\text{relaxation time scale of undamaged ice} \\
    \eta_0&= E_0\lambda_{0}^{-1}
    \quad\text{apparent viscosity of undamaged ice in N\,m$^{-2}$\,s} \\
    C   &\approx 40,000\text{ N\,m$^{-2}$} \quad\text{cohesion}\\
    \nu &= 0.3 \quad\text{Poisson ratio}\\
    \mu &= 0.7 \quad\text{internal friction coefficient} \\
    q &= \left[\sqrt{\mu^{2}+1}+\mu\right]^{2}
    \quad\text{slope of M-C yield curve in principle stress space}\\
    s &= \frac{q-1}{q+1}
    \quad\text{slope of M-C yield curve in stress invariant space}\\
    \sigma_c &=\frac{2C\,H}{\sqrt{\mu^{2}+1}-\mu}
    \quad\text{uniaxial (unconfined) compressive strength}  \\
    &= 2C\,H\left[\sqrt{\mu^{2}+1}+\mu\right]
    =2C\,H\sqrt{q} \nonumber \\
    \sigma_t&=-\frac{\sigma_c}{q} = -\frac{2C\,H}{\sqrt{q}}
    \quad\text{tensile strength cut-off}\\
    d_{\mathrm{crit}} &= \min\left[1, \frac{\sigma_t}{\sigma_1},
      \frac{\sigma_c}{\sigma_2-q\,\sigma_1}\right]
    \quad\text{critical damage parameter} \\
    \eta_v(t) &= \eta_0\, d^{-\alpha}\,H\,\exp\left[-C^*(1-c)\right] \\
    \lambda^{-1}(t) &= \tau_r(t) = \frac{\eta_0}{E_0}\,d^{\alpha-1}, \quad
    \lambda(t) = \frac{E_0}{\eta_0}\,d^{1-\alpha}
  \end{alignat}

\item We exploit the analogy (\ref{eq:p0})-(\ref{eq:mu}) to use existing VP-code and turn it into MEB-code;
\item S/R \verb+seaice_calc_stress+: calculate VP stress from coefficients and strain rates $\sr_{ij}$
\item S/R \verb+seaice_meb_update_sigma+: update global variables sigPlus, sigMinus, sigma12, based on 
\item S/R \verb+seaice_meb_calc_stress+: compute new stress at time level $n$ from $\sigma_{ij}^{n-1}$ and $\sr_{ij}^{n}$, neglecting advection of stress, based on ($\beta=1$: implicit; $=0$: explicit time stepping):
  \begin{align}
    \frac{\bm{\sigma}^{n}-\bm{\sigma}^{n-1}}{\Delta{t}}
    +\lambda^{n}\left\{\beta\bm{\sigma}^{n}
    +(1-\beta)\bm{\sigma}^{n-1}\right\}
    &= E^{n} (\bm{K}:\srb^{n}) - \text{advection} = \dot{\bm{\sigma}}^{n}
  \end{align}
  \begin{align}
    \label{eq:discretizedstress}
    \leftrightarrow\bm{\sigma}^{n} &= \frac{\bm{\sigma}^{n-1}
                      \left\{1-(1-\beta)\Delta{t}\lambda^{n}\right\}
                      + \Delta{t}\,\dot{\bm{\sigma}}^{n}}
                                     {1+\beta\Delta{t}\lambda^{n}},
                                     \quad\left(\lambda^{n} =
                                     \frac{E^{n}}{\eta_{v}^{n}}\right) \\
    \intertext{drop explicit/implicit factor $\beta$}
%    \lambda^n &= \frac{E^n}{\eta_{v}^{n}} \\
    \bm{\sigma}^n &= \frac{1}{1+\lambda^{n}} \left[E^n\Delta{t}\,\bm{K}:\srb^n
                 + \bm{\sigma}^{n-1}\right] \\
    \intertext{from the previous timestep:}
    \bm{\sigma}^{n-1} &= \frac{1}{1+\lambda^{n-1}}\left[
                     E^{n-1}\Delta{t}\,
                     \bm{K}:\srb^{n-1} + \bm{\sigma}^{n-2}\right]
    \intertext{so}
    \bm{\sigma}^{n} &= \frac{1}{1+\lambda^{n}}
                   \biggl[E^{n}\Delta{t}\,\bm{K}:\srb^n
                   + \frac{1}{1+\lambda^{n-1}}
                   \biggl[E^{n-1}\Delta{t}\,\bm{K}:\srb^{n-1}
                   + \bm{\sigma}^{n-2}\biggr]\biggr] \\
                &= \begin{gathered}[t]
                  \frac{1}{1+\lambda^{n}}
                  \biggl[E^{n}\Delta{t}\,\bm{K}:\srb^n +
                  \frac{1}{1+\lambda^{n-1}} \biggl[
                  E^{n-1}\Delta{t}\,\bm{K}:\srb^{n-1} \ldots \\
                  + \frac{1}{1+\lambda^{n-2}} \biggl[
                   E^{n-2}\Delta{t}\,\bm{K}:\srb^{n-2} + \dots
                   \biggr]\biggr]\biggr]
                 \end{gathered}
  \end{align}
  so that (\ref{eq:lambda}) and (\ref{eq:zeta}) become:
  \begin{align}
    \label{eq:lambdaImpEx}
    \frac{E\lambda}{1+\beta\Delta{t}\lambda(t)}
                                   &\rightarrow \zeta - \eta \\
    \label{eq:muImpEx}
    \frac{EM}{1+\beta\Delta{t}\lambda(t)}&\rightarrow \eta \\
    \label{eq:zetaImpEx}\zeta &\leftarrow
                                \frac{E(\Lambda+M)}
            {1+\beta\Delta{t}\lambda(t)} \\
    \intertext{and (\ref{eq:mebplus}) and (\ref{eq:mebminus})}
    \label{eq:mebplusImpEx}
    \frac{E\left(\bm{K}:\srb\right)_{+}}{1+\beta\Delta{t}\lambda(t)}
                                   &= 2\,\zeta\,\sr_{+}\\
    \label{eq:mebminusImpEx}
    \frac{E\left(\bm{K}:\srb\right)_{-}}{1+\beta\Delta{t}\lambda(t)}
                                   &= 2\, \eta\,\sr_{-}
  \end{align}
  The first part of eq.\,(\ref{eq:discretizedstress}) \[\frac{\bm{\sigma}^{n-1}                     \left\{1-(1-\beta)\Delta{t}\lambda(t)\right\}}                  {1+\beta\Delta{t}\lambda(t)}\] is added to the rhs of the momentum equations
\item S/R \verb+seaice_recip_relaxtime+: compute $\lambda(t) = \frac{E(t)}{\eta_{v}(t)}$.
\item S/R \verb+seaice_meb_update_rhs+: update right hand side of momentum equations with stress divergence of previous time level $\nabla\frac{\bm{\sigma}^{n-1}[1-(1-\beta)\Delta{t}\lambda(t)]}{1+\beta\Delta{t}\lambda(t)}$
\item S/R \verb+seaice_calc_viscosity+: replace definition of $\zeta$, $\eta$, and $P$ with the appropriate forms (\ref{eq:p0})-(\ref{eq:mu}); include time stepping part $\frac{1}{1+\Delta{t}\lambda(t)}$ as in (\ref{eq:muImpEx}) and (\ref{eq:zetaImpEx})
\item S/R \verb+seaice_update_damage+: determine failure criteria and step damage equation in time 
\item averaging: $\eta$ needs to be averaged to $Z$-points ($\overline{\eta}^{Z}$) to compute $\sigma_{12}$, $\sigma_{12}$ and $\sr_{12}$ need to be averaged to $C$-points ($\overline{\sigma}_{12}^{C}$ and $\overline{\sr}_{12}^{C}$) to compute stress criteria and diagnostics
  \begin{align}
    \label{eq:sigma12one}
    \overline{\sigma}_{12,ij}^{C} &= \frac{\sigma_{12,ij}+\sigma_{12,i+1,j}
                                    +\sigma_{12,i,j+1}+\sigma_{12,i+1,j+1}}{4}\\
                                  &= \frac{2\,\overline{\eta}^Z_{ij}\sr_{12,ij}
                                    +2\,\overline{\eta}^Z_{ij}\sr_{12,i+1,j}
                                    +2\,\overline{\eta}^Z_{ij}\sr_{12,i,j+1}
                                    +2\,\overline{\eta}^Z_{ij}\sr_{12,i+1,j+1}}{4}\\
    \intertext{or}
    \label{eq:sigma12onesq}
    \overline{\sigma}_{12,ij}^{C} &= \sqrt{\frac{\sigma_{12,ij}^{2}
                                    +\sigma_{12,i+1,j}^{2}
                                    +\sigma_{12,i,j+1}^{2}
                                    +\sigma_{12,i+1,j+1}^{2}}{4}}\\
    \intertext{or}
    \label{eq:sigma12two}
    \overline{\sigma}_{12,ij}^{C} &= 2\,\eta\,\frac{
                                      \sr_{12,ij}+\sr_{12,i+1,j}
                                      +\sr_{12,i,j+1}+\sr_{12,i+1,j+1}}{4}
    \intertext{or}
    \label{eq:sigma12twosq}
    \overline{\sigma}_{12,ij}^{C} &= 2\,\eta\,\sqrt{\frac{
                                      \sr_{12,ij}^2+\sr_{12,i+1,j}^2
                                      +\sr_{12,i,j+1}^2+\sr_{12,i+1,j+1}^2}{4}}
  \end{align}
the latter two (\ref{eq:sigma12two}) and (\ref{eq:sigma12twosq}) are preferred for stress evaluation when computing damage because it involves less averaging, but it is not clear if it is better or more consistent than (\ref{eq:sigma12one}). Average of the squares is energy consistent?
\end{itemize}

\paragraph{Timestepping}
This is how it is currently done in the MITgcm sea ice model:
\begin{enumerate}
\item solve momentum equations for ice velocitis $\bm{u}^{n}$ with stress computed from eq.(\ref{eq:discretizedstress}):
  \[\bm{\sigma}^{n} = \frac{\bm{\sigma}_{M}^{n-1}
      + \Delta{t}\,E^{n-1} (\bm{K}:\srb^{n})}
    {1+\Delta{t}\lambda^{n-1}} = \frac{1}{1+\Delta{t}\lambda^{n-1}}\left\{
      \Delta{t}\,E^{n-1} (\bm{K}:\srb^{n}) + \bm{\sigma}_{M}^{n-1}\right\}
  \]
  (S/Rs \verb+seaice_calc_viscosities+, \verb+seaice_meb_update_rhs+) Note, that $\srb^{n}$ is from the current time level, implying an implicit treatment of the momentum equations; $\sigma_M^{n-1}$ is stored in the variables \verb+seaice_sigma1/2/12+;
\item S/R \verb+seaice_meb_calc_stress+: diagnose stress from new ice velocities (strain rates) using the same equation;
\item S/R \verb+seaice_update_damage+: compute critical damage $\Psi=\frac{\Delta{t}}{\tau_{d}}
  \left\{\frac{\tau_{d}}{\Delta{t}} + d_{\mathrm{crit}}^{n}-1\right\}
  \approx d_{\mathrm{crit}}^{n}$ if the stress is over-critical and step damage equations (explicit time stepping):
  \begin{align}
    \label{eq:damagestepping}
    d^{n} &= d^{n-1}\left\{1
            + \frac{\Delta{t}}{\tau_{d}}(d_{\mathrm{crit}}-1)\right\}
            + \frac{\Delta{t}}{\tau_{h}} \\
            \intertext{without healing ($\tau_h=\infty$), this is the same as:}
    \nonumber d^{n}&=d^{n-1}\Psi \\ 
           \intertext{and with $\tau_d=\Delta{t}$} 
    \nonumber &=d^{n-1}\,d_{\mathrm{crit}}
  \end{align}
\item adjust/recompute stress to satisfy yield criteria and store stress:
  \begin{align}
    \bm{\sigma}_{M}^{n} &= \bm{\sigma}^{n}\,\Psi \\
                       &= \frac{\Psi\bm{\sigma}_{M}^{n-1}
                          + \Delta{t}\,E^{n} (\bm{K}:\srb^{n})}
                          {1+\Delta{t}\lambda^{n-1}}
  \end{align}
  because $E^{n}= d_{\mathrm{crit}} E^{n-1} \approx \Psi E^{n-1}$.
\item (possibly) iterate 1.-5.\ until convergence \`a la \citet{dansereau16:_meb}.
\item step advection of thickness, concentration, etc., including damage $d$
\item step thermodynamic equations (S/R \verb+seaice_growth+)
\end{enumerate}
Problems:
\begin{itemize}
\item $\sigma_{12}$ is not co-located with the remaining variables and averaging is required. This can lead to a mismatch between the stress that is used to evaluate the critical damage ($\sigma_{12}$ averaged to C-points) and the stress that is used in computing the ice velocities ($\Psi$ averaged to Z-points). This increases the numerical stencil and also leads to excessive smoothing, if not done properly.
\item defined at C-points: $\Psi$, $\lambda$, $E$, $\sigma_{11}$, $\sigma_{22}$, $\sigma_{1}$, $\sigma_{2}$
\item defined at Z-points: $\sigma_{12}$, $\sr_{12}$
\item averaging between $\sigma_{12}^C$ (needed for computing $\Psi$) and $\sigma_{12}^Z$ (needed for computing $\bm{\nabla\cdot\sigma}$):
  \begin{align}
    \sigma_{12}^{n,C} &= \frac{1}{1+\Delta{t}\lambda^{n-1}}
                      \left\{\frac{\Delta{t}E^{n-1}}{1+\nu}
                      \,\overline{\sr_{12}^{n}}^{C}
                      +\overline{\sigma_{M,12}^{n-1}}^{C}\right\} \\
    \sigma_{M,12}^{n,C} &= \sigma_{12}^{n,C}\,\Psi \\
    \sigma_{M,12}^{n,Z} &= \frac{1}{1+\Delta{t}\overline{\lambda^{n-1}}^{Z}}
                          \left\{\frac{\Delta{t}\overline{E^{n-1}}^{Z}}{1+\nu}\,
                          \sr_{12}^{n}
                        + \overline{\sigma_{M,12}^{n,C}}^Z\right\}; \\
    \intertext{If we store the strain rates $\srb_M$ instead of the stresses, we can write:}
    \label{eq:sigma12nC}
    \overline{\sigma_{12}^{n}}^C &= \frac{1}{1+\Delta{t}\lambda^{n-1}}
                      \left\{\frac{\Delta{t}E^{n-1}}{1+\nu}
                      \,\overline{\sr_{12}^{n}}^{C}
                      +\frac{1}{1+\Delta{t}\lambda^{n-1}}\left\{
                      \frac{\Delta{t}E^{n-1}}{1+\nu}
                      \,\overline{\sr_{M,12}^{n-1,Z}}^{C}\right\}\right\} \\
    \label{eq:sigma12n}
    \sigma_{12}^{n} &= \frac{1}{1+\Delta{t}\overline{\lambda^{n-1}}^{Z}}
                      \left\{\frac{\Delta{t}\overline{E^{n-1}}^{Z}}{1+\nu}
                      \,\sr_{12}^{n}
                      +\frac{\overline{\Psi}^Z}
                      {1+\Delta{t}\overline{\lambda^{n-1}}^{Z}}
                      \left\{
                      \frac{\Delta{t}\overline{E^{n-1}}^{Z}}{1+\nu}
                      \,\sr_{M,12}^{n-1}
                      \right\}
                      \right\}\\
    \sr_{M,12}^{n} &= \sr_{12}^{n}
                   +\frac{\sr_{M,12}^{n-1}}{1+\Delta{t}\overline{\lambda^{n-1}}^Z}
  \end{align}
  and we may get away with less averaging. Eq.\,(\ref{eq:sigma12nC}) is used to evaluate the principle stresses at C-points and eq.\,(\ref{eq:sigma12n}) is used to compute the new velocity field from $\bm{\nabla\cdot\sigma}$.
\end{itemize}


\end{document}
%%% Local Variables:
%%% mode: latex
%%% TeX-master: t
%%% End:
